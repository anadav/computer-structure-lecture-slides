
\documentclass[aspectratio=169]{beamer}
\usepackage{tikz}
\usepackage{array}
\usepackage{enumitem}
\usepackage{xcolor}
\usepackage{amsmath}
\usetikzlibrary{automata,positioning,arrows.meta,backgrounds}
\usetikzlibrary{arrows.meta,positioning,calc,matrix}

% Define colors
\definecolor{myblue}{RGB}{0,0,255}
\definecolor{mypurple}{RGB}{128,0,128}
\definecolor{mygreen}{RGB}{0,128,0}
\definecolor{myorange}{RGB}{255,140,0}
\definecolor{lightblue}{RGB}{173,216,230}
\definecolor{peach}{RGB}{255,218,185}
\definecolor{lightpurple}{RGB}{230,230,250}

\usetheme{default}
\setbeamertemplate{navigation symbols}{}

\title{Computer Structure\\\large Out-Of-Order Execution}
\author{Lihu Rappoport}
\institute{Technion -- Computer Science}
\date{}

\setbeamertemplate{navigation symbols}{}
\setbeamertemplate{footline}[frame number]

\begin{document}

\begin{frame}
  \titlepage
\end{frame}

\begin{frame}
\frametitle{What's Next}
\begin{itemize}
  \item[$\diamond$] \textbf{Goal:} minimize CPU Time
  \begin{equation*}
    	ext{CPU Time} = \text{clock cycle} \times \text{CPI} \times \text{IC}
  \end{equation*}
  \item[$\diamond$] So far we have learned
  \begin{itemize}
    \item[$\triangleright$] Minimize clock cycle $\Rightarrow$ add more pipe stages
    \item[$\triangleright$] Minimize CPI $\Rightarrow$ use pipeline
    \item[$\triangleright$] Minimize IC $\Rightarrow$ architecture
  \end{itemize}
  \item[$\diamond$] In a pipelined CPU
  \begin{itemize}
    \item[$\triangleright$] CPI without hazards is 1
    \item[$\triangleright$] CPI with hazards is $> 1$
  \end{itemize}
  \item[$\diamond$] Adding more pipe stages reduces clock cycle \emph{but} increases CPI
  \begin{itemize}
    \item[$\triangleright$] Higher penalty due to control hazards
    \item[$\triangleright$] More data hazards
  \end{itemize}
  \item[$\diamond$] What can we do? \textbf{Further reduce the CPI!}
\end{itemize}
\end{frame}

\begin{frame}
\frametitle{A Superscalar CPU}
\begin{itemize}
  \item[$\diamond$] Duplicating hardware in one pipe stage won’t help
  \begin{itemize}
    \item[$\triangleright$] e.g., have 2 ALUs
    \item[$\triangleright$] the bottleneck moves to other stages
  \end{itemize}
  \item[$\diamond$] Getting IPC $>1$ requires fetching, decoding, executing, and retiring $>1$ instruction per clock
\end{itemize}
\end{frame}

\begin{frame}{The Pentium\textregistered{} Processor (1993)}
\begin{itemize}
  \item Fetch and decode 2 instructions per cycle
  \item At decode stage decide on \emph{pairing}: can the two instructions be executed in parallel
  \item Pairing decision is based on
  \begin{itemize}
    \item \textbf{Data dependencies:} 2\textsuperscript{nd} instruction must be independent of 1\textsuperscript{st}
    \begin{itemize}
      \item If the 1\textsuperscript{st} instruction produces data needed by the 2\textsuperscript{nd} instruction, the 2\textsuperscript{nd} instruction cannot be executed in parallel to the 1\textsuperscript{st}.
    \end{itemize}
    \item \textbf{Resources:} U-pipe and V-pipe are not symmetric (save HW)
    \begin{itemize}
      \item Common instructions can execute on either pipe
      \item Some instructions can execute only on the U-pipe
      \item V-pipe can run a subset of the instructions that can run on U-pipe
      \item If the 2\textsuperscript{nd} instruction requires the U-pipe, it cannot pair
      \item Some instructions use resources of both pipes (and therefore cannot pair)
    \end{itemize}
  \end{itemize}
\end{itemize}
\end{frame}

\begin{frame}{Misprediction Penalty in a Superscalar CPU}
\begin{itemize}
  \item Assume
  \begin{itemize}
    \item 1 / 100 instruction is a mispredicted jump (MPI $=1\%$)
    \item Flush penalty of 5 cycles
  \end{itemize}
  \item For IPC $=2$
  \begin{itemize}
    \item Flush every $100/2 = 50$ cycles
    \item 5 cycles flush penalty every 50 cycles $\Rightarrow$ $10\%$ performance hit
  \end{itemize}
  \item For IPC $=1$
  \begin{itemize}
    \item Flush every $100/1 = 100$ cycles
    \item 5 cycles flush penalty per 100 cycles $\Rightarrow$ $5\%$ performance hit
  \end{itemize}
  \item \textbf{Flush penalty increases as the machine is deeper and wider.}
\end{itemize}
\end{frame}

\begin{frame}{Extract More ILP}
\begin{itemize}
  \item \textbf{ILP -- Instruction Level Parallelism}
  \begin{itemize}
    \item A given program, executed on a given input data has a given parallelism
    \item Can execute only independent instructions in parallel
    \item If, for example, each instruction is dependent on the previous instruction, ILP $= 1$ (adding more HW will not change that)
  \end{itemize}
  \item Adjacent instructions are usually dependent
  \begin{itemize}
    \item 2\textsuperscript{nd} pipe utilization is lower than 1\textsuperscript{st}; a 3\textsuperscript{rd} pipe utilization further goes down
  \end{itemize}
  \item Cache misses stall execution -- instructions dependent on a missed load stall
  \item Compilers are limited to basic blocks, and are not aware of cache misses
  \item \textbf{Solution: Out-Of-Order Execution}
  \begin{itemize}
    \item Look for independent instructions further ahead in the program
    \item Execute instructions based on data readiness
    \item Still need to keep the semantics of the original program
  \end{itemize}
\end{itemize}
\end{frame}

\begin{frame}[fragile]{Data Flow Analysis --- Example}
\textbf{Assume divide takes multiple cycles.}
\vspace{0.5em}

\begin{verbatim}
(1) r1 <- r4 / r7
(2) r8 <- r1 + r2
(3) r5 <- r5 + 1
(4) r6 <- r6 - r3
(5) r4 <- r5 + r6
(6) r7 <- r8 * r4
\end{verbatim}

\begin{itemize}
  \item Out-of-order allows executing (3) and (4) while (1) divides.
  \item Once (3) and (4) complete, (5) can proceed, then (6).
\end{itemize}
\end{frame}

\begin{frame}{OOOE --- General Scheme}
\begin{itemize}
  \item \textbf{Fetch \& decode} instructions in parallel but in-order
  \begin{itemize}
    \item Fill the \emph{Instruction Pool}
  \end{itemize}
  \item \textbf{Execute} ready instructions from the instruction pool
  \begin{itemize}
    \item All source data ready $+$ needed execution resources available
  \end{itemize}
  \item Once an instruction is executed, signal all dependent instructions that data is ready
  \item \textbf{Commit} instructions in parallel but in-order
  \begin{itemize}
    \item State change (memory, register) and fault/exception handling
  \end{itemize}
\end{itemize}
\end{frame}

\begin{frame}[fragile]{Write-After-Write Dependency (1/2)}
\begin{verbatim}
(1) r1 <- r9 / 17
(2) r2 <- r2 + r1
(3) r1 <- 23
(4) r3 <- r3 + r1
(5) jcc L2
(6) L2: r1 <- 35
(7) r4 <- r3 + r1
(8) r3 <- 2
\end{verbatim}
\end{frame}

\begin{frame}{Write-After-Write Dependency (2/2)}
\begin{itemize}
  \item If instruction (3) is executed before instruction (1), \texttt{r1} ends up having a wrong value.
  \item Called \textbf{write-after-write} false dependency: the order of two writes to the same register is flipped.
\end{itemize}
\end{frame}

\end{document}
