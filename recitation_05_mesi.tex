\documentclass[aspectratio=169,12pt]{beamer}
\usepackage[utf8]{inputenc}
\usepackage{amsmath, amssymb}
\usepackage{booktabs}
\usepackage{colortbl}
\usepackage{hyperref}
\usepackage{makecell}
\usepackage{ragged2e}
\usepackage{bytefield}
\usepackage{tikz}
\usetikzlibrary{arrows.meta, positioning, shapes.geometric, calc, tikzmark, shapes.misc}
\usepackage{tcolorbox}
\usetheme{Madrid}
\title{MESI Protocol}
\subtitle{Computer Architecture}
\author{Course 234267}
\date{}
\begin{document}
\frame{\titlepage}

%==========================================
\begin{frame}{Multi-processor System}
\begin{block}{Memory System Coherence}
A memory system is coherent if:
\end{block}

\begin{enumerate}
\item If P1 writes to address X, and later on, P2 reads X, and there are no other writes to X in between
    \begin{itemize}
    \item[$\Rightarrow$] P2's read returns the value written by P1's write
    \end{itemize}
\item Writes to the same location are serialized:
    \begin{itemize}
    \item Two writes to location X are seen in the same order by all processors
    \end{itemize}
\end{enumerate}

\vspace{1em}
\begin{center}
% Placeholder for processor-cache-memory diagram
\framebox[0.6\textwidth][c]{
\parbox{0.55\textwidth}{\centering
[Diagram: Two processors with L1 caches\\
connected to shared L2 cache and memory]
}}
\end{center}
\end{frame}

%==========================================
\begin{frame}{MESI Protocol States}
Each cache line can be in one of 4 states:

\begin{itemize}
\item \textbf{Invalid (I)} -- Line's data is not valid
\vspace{0.5em}
\item \textbf{Shared (S)} -- Line is valid and not dirty, copies may exist in other processors
\vspace{0.5em}
\item \textbf{Exclusive (E)} -- Line is valid and not dirty, other processors do not have the line in their local caches
\vspace{0.5em}
\item \textbf{Modified (M)} -- Line is valid and dirty, other processors do not have the line in their local caches
\end{itemize}
\end{frame}

%==========================================
\begin{frame}{Multi-processor System: Example (1/3)}
\begin{columns}
\column{0.5\textwidth}
\textbf{Sequence of operations:}
\begin{itemize}
\item P1 reads 1000
\item P1 writes 1000
\end{itemize}

\vspace{1em}
\textbf{Initial state:}
\begin{itemize}
\item Memory[1000]: 5
\item Both caches empty
\end{itemize}

\column{0.5\textwidth}
\begin{center}
% Placeholder for state diagram
\framebox[0.9\columnwidth][c]{
\parbox{0.85\columnwidth}{\centering
[Diagram: P1 cache transitions\\
from Invalid $\rightarrow$ Exclusive $\rightarrow$ Modified\\
P1 L1: [1000]: 6 (M)\\
P2 L1: Invalid\\
L2: [1000]: 5, CVB: 001]
}}
\end{center}
\end{columns}
\end{frame}

%==========================================
\begin{frame}{Multi-processor System: Example (2/3)}
\begin{columns}
\column{0.5\textwidth}
\textbf{Continuing sequence:}
\begin{itemize}
\item P2 reads 1000
\item L2 snoops 1000
\item P1 writes back 1000
\item P2 gets 1000
\end{itemize}

\vspace{1em}
\textbf{Result:}
\begin{itemize}
\item Both P1 and P2 have line in Shared state
\item Value = 6
\end{itemize}

\column{0.5\textwidth}
\begin{center}
% Placeholder for state diagram
\framebox[0.9\columnwidth][c]{
\parbox{0.85\columnwidth}{\centering
[Diagram: State transitions\\
P1: M $\rightarrow$ S\\
P2: I $\rightarrow$ S\\
Both caches: [1000]: 6 (S)\\
L2: [1000]: 6, CVB: 101]
}}
\end{center}
\end{columns}
\end{frame}

%==========================================
\begin{frame}{Multi-processor System: Example (3/3)}
\begin{columns}
\column{0.5\textwidth}
\textbf{Final operation:}
\begin{itemize}
\item P2 requests for ownership with write intent
\end{itemize}

\vspace{1em}
\textbf{Result:}
\begin{itemize}
\item P1: Invalid
\item P2: Exclusive (then can modify)
\end{itemize}

\column{0.5\textwidth}
\begin{center}
% Placeholder for state diagram
\framebox[0.9\columnwidth][c]{
\parbox{0.85\columnwidth}{\centering
[Diagram: State transitions\\
P1: S $\rightarrow$ I\\
P2: S $\rightarrow$ E\\
P1 L1: Invalid\\
P2 L1: [1000]: 6 (E)\\
L2: CVB: 01]
}}
\end{center}
\end{columns}
\end{frame}

%==========================================
\begin{frame}{Core Valid Bits and Inclusion}
\begin{block}{Core Valid Bits (CVB)}
L2 keeps track of the presence of each line in each Core's L1 caches
\end{block}

\textbf{Purpose:}
\begin{itemize}
\item Determine if it needs to send a snoop to a processor
\item Determine in what state to provide a requested line (S, E)
\item Maintain Core Valid Bits (CVB) per cache line
\end{itemize}

\vspace{0.5em}
\begin{alertblock}{Inclusion Property}
Need to guarantee that the L1 caches in each Core are included in the L2 cache
\end{alertblock}

\textbf{When L2 evicts a line:}
\begin{itemize}
\item L2 sends a snoop invalidate to all processors that have it
\item If the line is modified in L1 (exists only in that processor):
    \begin{itemize}
    \item The processor responds by sending the updated value to L2
    \item When evicted from L2, the updated value gets written to memory
    \end{itemize}
\end{itemize}
\end{frame}

%==========================================
\begin{frame}{MESI Protocol States - Summary}
\begin{table}[h]
\centering
\begin{tabular}{|c|c|c|c|}
\hline
\rowcolor{blue!20}
\textbf{State} & \textbf{Valid} & \textbf{Modified} & \textbf{Copies may exist} \\
& & & \textbf{in other processors} \\
\hline
Invalid & No & N.A. & N.A. \\
\hline
Shared & Yes & No & Yes \\
\hline
Exclusive & Yes & No & No \\
\hline
Modified & Yes & Yes & No \\
\hline
\end{tabular}
\end{table}

\vspace{1em}
\begin{block}{Important Rule}
A modified line must be exclusive
\begin{itemize}
\item Otherwise, another processor with the line will be using stale data
\item Therefore, before modifying a line, a processor must request ownership
\end{itemize}
\end{block}
\end{frame}

%==========================================
\begin{frame}{MESI Protocol Example - 4 Processors}
\small
Four-processor shared-memory system with MESI protocol

\begin{table}[h]
\centering
\begin{tabular}{|l|c|c|c|c|c|}
\hline
\rowcolor{blue!20}
\textbf{Operation} & \textbf{P0} & \textbf{P1} & \textbf{P2} & \textbf{P3} & \textbf{CVBs} \\
\hline
Initial State & I & I & I & I & 0000 \\
\hline
P0 reads X & E & I & I & I & 1000 \\
\hline
P1 reads X & S & S & I & I & 1100 \\
\hline
P2 reads X & S & S & S & I & 1110 \\
\hline
P3 writes X & I & I & I & M & 0001 \\
\hline
P0 reads X & S & I & I & S & 1001 \\
\hline
\end{tabular}
\end{table}

\textbf{Key observations:}
\begin{itemize}
\item First reader gets Exclusive state (if no other copies exist)
\item Write operation invalidates all other copies
\item Subsequent reads after a write result in Shared state
\end{itemize}
\end{frame}

%==========================================
\begin{frame}{MESI Question 2 - Setup}
\textbf{System Configuration:}
\begin{itemize}
\item 3 Core processor with MESI
\item Each core has an L1 cache, L2 cache is shared
\end{itemize}

\textbf{Message Types:}
\begin{columns}
\column{0.5\textwidth}
\textbf{L1 $\rightarrow$ L2:}
\begin{itemize}
\item Read Address (A)
\item RFO (A) - Request for Ownership
\item Data (A)
\end{itemize}

\column{0.5\textwidth}
\textbf{L2 $\rightarrow$ L1/Memory:}
\begin{itemize}
\item Read Address (A) to memory
\item Data (A) to L1 with MESI state
\item RFO (A) - after RFO from L1
\item Snoop (A) to L1
\end{itemize}
\end{columns}

\vspace{1em}
\textbf{Timing:}
\begin{itemize}
\item L1 $\leftrightarrow$ L2: 10ns
\item L2 $\leftrightarrow$ Memory: 100ns
\end{itemize}
\end{frame}

%==========================================
\begin{frame}{MESI Question 2 - Solution}
\tiny
\begin{table}[h]
\centering
\begin{tabular}{|p{1.5cm}|c|c|c|c|c|c|c|p{5cm}|}
\hline
\rowcolor{blue!20}
\textbf{Operation} & \textbf{P1} & \textbf{P2} & \textbf{P3} & \multicolumn{3}{c|}{\textbf{CVBs}} & \textbf{Time} & \textbf{Messages} \\
\cline{5-7}
& & & & P1 & P2 & P3 & & \\
\hline
Initial & I & I & I & 0 & 0 & 0 & 0 & - \\
\hline
P2 reads A & I & E & I & 0 & 1 & 0 & 220 & L1 miss/req to L2, L2 Miss/req to Mem, Mem data to L2, L2 data to L1 \\
\hline
P2 writes A & I & M & I & 0 & 1 & 0 & 0 & No messages (local operation) \\
\hline
P1 reads A & S & S & I & 1 & 1 & 0 & 40 & P1: miss/req to L2, P2: L2→L1 snoop, P2: L1 to L2 data, P1: L2 to L1 data \\
\hline
P2 reads A & S & S & I & 1 & 1 & 0 & 0 & Hit in cache \\
\hline
P3 reads A & S & S & S & 1 & 1 & 1 & 20 & P3: L1 miss/Req to L2, L2 to P3 data \\
\hline
P3 writes A & I & I & M & 0 & 0 & 1 & 30 & P3: L1 to L2 RFO, L2 snoop P1+P2, L2 to P3 RFO granted \\
\hline
\end{tabular}
\end{table}
\normalsize
\end{frame}

%==========================================
\begin{frame}{Read For Ownership (RFO)}
\begin{block}{RFO Request}
A signal from private to shared cache (L1$\rightarrow$L2) requesting cache line exclusivity for write intent
\end{block}

\textbf{RFO Process:}
\begin{enumerate}
\item L1 sends RFO to L2/LLC
\item MLC/LLC invalidates cache line in other L1s
\item MLC/LLC responds to L1 that RFO has been granted
\item L1 can now modify cache line
\end{enumerate}

\vspace{1em}
\begin{alertblock}{Key Point}
RFO ensures exclusive access before modification, maintaining cache coherence
\end{alertblock}
\end{frame}

%==========================================
\begin{frame}{Global Observation (GO)}
\begin{block}{Global Observation}
Before sending actual data to L1, L2 responds that the line is observed to be in it by all other processors
\end{block}

\textbf{The Global Observation carries the MESI state: E/S}

\vspace{1em}
\textbf{Two-step L1$\rightarrow$L2 line request:}
\begin{enumerate}
\item \textbf{GO} - Global Observation with MESI state
\item \textbf{Fill} - Actual data transfer
\end{enumerate}

\vspace{1em}
\begin{alertblock}{Note}
GO and data may be sent at different cycles:
\begin{itemize}
\item GO is the outcome of a tag-hit
\item Data comes from the data array
\end{itemize}
\end{alertblock}
\end{frame}

%==========================================
\begin{frame}{MESI Question 3 - Exam Problem Setup}
\footnotesize
\textbf{System Configuration:}
\begin{itemize}
\item 2 processors (P0 and P1) with MESI protocol
\item Each processor has L1 cache, shared L2 cache (inclusive)
\item Direct-mapped caches with write allocate + write back
\end{itemize}

\textbf{Latencies:}
\begin{itemize}
\item L1 access: 3ns
\item L1 $\leftrightarrow$ L2 message: 20ns
\item L2 $\leftrightarrow$ Memory message: 90ns
\end{itemize}

\textbf{Important Details:}
\begin{itemize}
\item Parallel messages possible on different channels
\item X1-X5 are separate cache lines
\item X1-X4 map to same L1 set, X5 to different L1 set
\item X1-X2 map to same L2 set, X3-X5 to different L2 set
\end{itemize}
\end{frame}

%==========================================
\begin{frame}{Ring Interconnect}
\begin{center}
% Placeholder for ring interconnect diagram
\framebox[0.8\textwidth][c]{
\parbox{0.75\textwidth}{\centering
[Diagram: Ring Interconnect Architecture\\
Cores 0-3 with LLC slices\\
System Agent, Graphics, IMC, DMI, PCI Express\\
Connected in ring topology]
}}
\end{center}
\end{frame}

%==========================================
\begin{frame}{Ring Architecture Details}
\textbf{Ring Configuration:}
\begin{itemize}
\item 2 × 4 rings: Req / Data / Ack / Snoop
\item Packets always use the shortest path
\item Static Even/Odd polarity per station
\item Each ring switches polarity on each cycle
\end{itemize}

\vspace{1em}
\begin{block}{Key Mechanism}
A stop can only pull data from the direction (ring) that matches its polarity in the current cycle
\begin{itemize}
\item[$\Rightarrow$] Sender must ensure data can be pulled by receiver when it arrives
\end{itemize}
\end{block}

\textbf{Example Configuration:}
\begin{itemize}
\item 64B cache line (L1/L2/LLC)
\item 32B/cycle bus bandwidth
\item[$\Rightarrow$] Cache line transfers from LLC to L1 in two strokes
\end{itemize}
\end{frame}

%==========================================
\begin{frame}{Ring Example - Data Transfer}
\footnotesize
\textbf{Scenario:} Core 0, Core 1, and GFX issue data read requests

\vspace{0.5em}
\begin{center}
% Placeholder for ring timing diagram
\framebox[0.9\textwidth][c]{
\parbox{0.85\textwidth}{\centering
[Timing Diagram: Ring cycles 1-15\\
Shows Request, Global Observation, and Data phases\\
Even/Odd polarity switching per cycle]
}}
\end{center}

\vspace{0.5em}
\textbf{Key Events:}
\begin{enumerate}
\item GFX request delayed (odd distance, must arrive on Even cycle)
\item Both Core requests hit in LLC, CVBs indicate no snoop needed
\item LLC sends first chunk (1/2 cache line) to each core
\item LLC sends GO with MESI state (E/S)
\item LLC sends second chunk to complete transfer
\end{enumerate}
\end{frame}

%==========================================
\begin{frame}{Summary}
\begin{itemize}
\item \textbf{MESI Protocol} ensures cache coherence in multi-processor systems
\vspace{0.5em}
\item \textbf{Four States:} Invalid, Shared, Exclusive, Modified
\vspace{0.5em}
\item \textbf{Key Mechanisms:}
    \begin{itemize}
    \item Core Valid Bits (CVB) for tracking
    \item Read For Ownership (RFO) for write intent
    \item Global Observation (GO) for state notification
    \end{itemize}
\vspace{0.5em}
\item \textbf{Ring Interconnect:} Efficient on-chip communication with polarity-based routing
\vspace{0.5em}
\item \textbf{Performance:} Message latencies crucial for overall system performance
\end{itemize}
\end{frame}

\end{document}