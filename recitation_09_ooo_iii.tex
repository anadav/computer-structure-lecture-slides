\documentclass[aspectratio=169,12pt]{beamer}
\usepackage[utf8]{inputenc}
\usepackage{amsmath, amssymb}
\usepackage{booktabs}
\usepackage{colortbl}
\usepackage{multirow}
\usepackage{hyperref}
\usepackage{makecell}
\usepackage{ragged2e}
\usepackage{tikz}
\usetikzlibrary{arrows.meta, positioning, shapes.geometric, calc, tikzmark, shapes.misc, fit, decorations.pathreplacing, matrix}
\usepackage{tcolorbox}
\usepackage{array}
\usepackage{listings}
\usepackage{pgfkeys}
\usepackage{adjustbox}
\usepackage[normalem]{ulem} 
\usetheme{Madrid}

\title{Out-of-Order Execution: Part III}
\subtitle{Example Problem}
\author{Computer Architecture 234267}
\date{2025, Recitation \#9}

\begin{document}

\begin{frame}[fragile]
\frametitle{P6 $\mu$Architecture - Instruction Execution Timeline}

\begin{adjustbox}{width=\textwidth,center}
\footnotesize
\begin{tabular}{|c|l|c|c|c|c|c|c|c|c|c|c|c|c|c|c|c|}
\hline
\multirow{2}{*}{\#} & \multirow{2}{*}{Instruction} & \multicolumn{3}{c|}{Reg Values} & \multicolumn{2}{c|}{Mem} & \multicolumn{9}{c|}{Execution Timeline} \\
\cline{3-17}
 & & R1 & R2 & R3 & addr & data & \rotatebox{90}{T alloc} & src1 & src2 & Imm & \rotatebox{90}{T src1 rdy} & \rotatebox{90}{T src2 rdy} & \rotatebox{90}{T exe} & \rotatebox{90}{Load block} & \rotatebox{90}{T data rdy} & \rotatebox{90}{T commit} \\
\hline
    \onslide<2->{0 & \texttt{ld R2=[R1+30]} & 10 & 40 & - & 40 & 40 & 1 & R1 & - & 30 & 1 & - & 2 & 0 & 11 & 12} \\
    \onslide<3->{1 & \texttt{st [R2+20]=R1} & - & 60 & - & 60 & 10 & 1 & P0 & R1 & 20 & 11 & 1 & \makecell{\tiny Std:2\\[-4pt]\tiny Sta:12} & - & 12 & 13} \\
    \onslide<4->{2 & \texttt{ld R3=[R1+100]} & - & - & 110 & 110 & 110 & 1 & R1 & - & 100 & 1 & - & 2 & 1 & 21 & 22} \\
    \onslide<5->{3 & \texttt{st [R1+40]=R3} & - & - & - & 50 & 110 & 1 & R1 & P2 & 40 & 1 & 21 & \makecell{\tiny Std:22\\[-4pt]\tiny Sta:2} & - & 22 & 23} \\
    \onslide<6->{4 & \texttt{add R1=R1+10} & 20 & - & - & - & - & 2 & R1 & - & 10 & 2 & - & 3 & - & - & 23} \\
    \onslide<7->{5 & \texttt{blt (R1<100)} & - & - & - & - & - & 2 & P4 & - & 100 & 3 & - & 4 & - & - & 23} \\
    \hline
    \onslide<8->{6 & \texttt{ld R2=[R1+30]} & - & 50 & - & 50 & 110 & 2 & P4 & - & 30 & 3 & - & 4 & 1,2 & 24 & 25} \\
    \onslide<9->{7 & \texttt{st [R2+20]=R1} & - & - & - & 130 & 20 & 2 & P6 & P4 & 20 & 24 & 3 & \makecell{\tiny Std:4\\[-4pt]\tiny Sta:25} & - & 25 & 26} \\
    \onslide<10->{8 & \texttt{ld R3=[R1+100]} & - & - & 120 & 120 & 120 & 3 & P4 & - & 100 & 3 & - & 4 & 1 & 27 & 28} \\
    \onslide<11->{9 & \texttt{st [R1+40]=R3} & - & - & - & 60 & 120 & 3 & P4 & P8 & 40 & 3 & 27 & \makecell{\tiny Sta:4\\[-4pt]\tiny Std:28} & - & 28 & 29} \\
    \onslide<12->{10 & \texttt{add R1=R1+10} & 30 & - & - & - & - & 3 & P4 & - & 10 & 3 & - & 4 & - & - & 29} \\
    \onslide<13->{11 & \texttt{blt (R1<100)} & - & - & - & - & - & 3 & P10 & - & 100 & 4 & - & 5 & - & - & 29} \\
    \hline
\end{tabular}
\end{adjustbox}

\vspace{0.3cm}
\onslide<14->{
\begin{columns}[T]
\begin{column}{0.48\textwidth}
\scriptsize
\textbf{Register Renaming:}
\begin{itemize}
\item Pi: Physical register i
\item Ri: Architectural register i
\end{itemize}
\end{column}
    \begin{column}{0.48\textwidth}
    \scriptsize
    \textbf{Store Load Block Codes:}
        \begin{itemize}
        \item 0: ready
        \item 1: address blocking
        \item 2: data not ready
        \end{itemize}
    \end{column}
\end{columns}
}
\end{frame}

\end{document}